\section{はじめに}

本文書はFPGAやASIC設計を目的としたRTL設計の際に使用するEDA (Electronic design automation) ツールを,Dockerを用いて従来より簡単に運用するためのハウツーを記しております.
近年のRTL設計では開発規模の影響により複数人によって開発設計されることが多くなってきました.
従来のローカル環境へ設計者自身がツールをインストールすることは,設計者の時間を無駄にすると共に,ツールのバージョンを揃えるためにOSをダウングレードするなど,現在の開発環境に適したとはいえない運用です.
一方で,EDA用に計算資源の豊富なサーバを用意し,環境を構築するという手段は,集中管理によってツールのバージョンを揃えることができる点が一見優れているように見えますが,実際の開発ではレガシーな環境を維持しなくてはならないため,別途レガシーな環境を残した実機を保存するといったメンテナンス性の低い対応がとられております.
また,近年の計算機の進歩から開発者が使用する計算機の資源を活用した方が,計算資源を有効活用できると言えます.

そこで,2013年に登場した新たな仮想化技術であるDockerを用いて,OSに依存せずにEDAツールをローカル環境で利用できるようにします.
Dockerは従来の仮想化技術であるホスト型やハイパーバイザ型よりファイルサイズおよび遅延時間を小さく抑えながら運用することができます.
また,Dockerはコンテナとして仮想イメージを扱い,コンテナはアプリケーションごとに管理されます.
これにより,レガシーな環境をコンテナとして扱うことで,他の環境と共存が可能です.
また,Dockerの設定ファイルをRTLと同様にプロジェクトへ含めることで,設計成果物と設計環境を紐付けることができ,開発環境,テスト環境,運用環境を統一することができます.

副次効果として,DockerはLinux環境を仮想化することになりますので,RTL開発をLinuxベースで行うこととなります.
Linuxの優れた点の1つとして,プログラミング環境の構築がパッケージ管理システムを利用することで平易にできるという点が挙げられます.
これにより,RTL開発をサポートする環境をより整えることができます.

本文書では以下の話題について取り扱います.

\begin{itemize}
  \item DockerによるEDA環境構築
  \item Linuxによるサポートを利用したRTL設計
\end{itemize}
