\section{Dockerを用いたEDA環境の構築}

本節では\ac{eda}環境の構築について記載します.
\ac{eda}環境はDockerを用いて管理します.
Dockerを使う利点として以下の3点が挙げられます.
\begin{itemize}
  \item \ac{eda}ツールのシステム要求から解放される.
  \item \ac{eda}ツールのインストールの手間が省ける.
  \item 開発環境の要求仕様がコードとして明文化できる.
\end{itemize}
それぞれの利点について説明します.
まず1点目ですが,Dockerが動作する環境であれば,WindowsやLinuxといったOSの差異に囚われることなくEDAツールのが選択できます.
また,(なるべくないことが望ましいのですが)レガシーなツールを維持しなければならない場合,Dockerイメージとして管理しておくことで,計算機上の開発環境を残しておく必要がなくなります.
ただし,外部のライブラリやハードウェアに依存する場合は注意が必要です.
Dockerイメージをリビルドした場合,開発環境が再現できない状況の発生がありえます.
また,対応OSが古すぎてDockerイメージが存在しない場合,自作する必要があるかもしれません.
\footnote{Dockerのベースとなるイメージを作るには以下を参照するとよいかもしれません: \url{https://docs.docker.com/develop/develop-images/baseimages/}}

2点目は開発者にとって重要です.
開発に携わる人々が単独もしくは複数どちらでも,\ac{eda}ツールをインストールした環境は複数必要となり,大抵の場合テスト用途と本番 (もしくは実機) 用途となるでしょう.
また,1点目と同様にそれぞれの環境がバラバラな可能性は大いにあり得ます.
そのため,使用する\ac{eda}ツールのバージョンの統合に難儀することがあります.
単純なインストール作業でも多くの時間を費やす一方で,新しくて速い計算機と優秀な人材が来た場合,\ac{eda}ツールのインストールという最低限の開発環境の構築で消耗していては,時間の浪費に他なりません.
Dockerを使えば,\ac{eda}ツールのバージョンを統一しやすくなるとともに,インストール手順もより簡略化できます.

3点目ですが,各人の開発環境に依存した成果物が構成された場合,設計の正しさをどうやって保証すれば良いのでしょうか.
\footnote{「私の手元では動くのだけれどね...」}
完全に保証することは不可能ですが,「開発環境も設計の一部である」と考えて,設計と同時に開発環境も管理した方が,成果物と開発環境の依存関係をシンプルに保ちやすくなるといえます.
Dockerを用いることで,常に設計物と依存する環境 (例えばツールのバージョンやプログラミング言語のバージョンなど) を紐付けておくことができます.
これによって開発環境の管理がより明確になります.
また,Dockerによって成果物以外のデータとの依存関係を断ち切ることで,成果物の冪等性を保ちます.
\footnote{冪等性とは,何度同じ操作をしたとしても同じ結果を得られるということです.}


また,本文書で扱う\ac{eda}ツールは以下の通りです.
\begin{itemize}
  \item ModelSim (Mentor Graphics): \ac{rtl}シミュレータ
  \item Quartus Prime (Intel FPGA): \ac{fpga}コンパイラ
\end{itemize}
